\chapter{Einleitung}

Anfang der 2000er Jahre sollte der Professor Massimo Banzi seinen Studenten beibringen, wie man interaktive elektrische Schaltungen für künstlerische Projekte erstellt. Leider erforderten die damals vorhandenen Mikrocontroller einiges an Hintergrundwissen, bevor man irgendetwas mit ihnen anfangen konnte. Professor Banzi hatte dieses Wissen - er mochte seinen Studenten, die ein künstlerisches Designstudium gewählt hatten, jedoch kein Studium zum Elektroingenieur zumuten, ehe sie fähig wären, künstlerische elektronische Projekte umzusetzen.

So kam es, dass Massimo Banzi und David Cuartielles im Jahr 2005 den ersten Arduino entwickelten. Dieser sollte einfach zu handhaben, günstig anzuschaffen und - eine ziemlich neue Idee für Hardware - sein Aufbau sollte frei zugänglich sein, sodass er auch von anderen nachgebaut werden konnte. Natürlich sollte auch die Entwicklungsumgebung, mit der sich der Arduino programmieren lässt, frei verfügbar sein. Diese drei Eigenschaften des Arduino führten innerhalb weniger Jahre zu einer unglaublichen Verbreitung des Mikrocontrollers, die heute nicht nur Studenten und Universitäten betrifft, sondern auch Bastler, Künstler und Schulen weltweit. Elektronik und Programmierung wurde von einer kleinen Nische für Nerds zu einem allgemein verfügbaren Werkzeug, mit dem jeder, der sich ein wenig mit dem Thema beschäftigt, seine Kreativität auf eine neue Weise ausleben kann.

Diese Sichtweise auf den Arduino ist ein wichtiges Ziel dieses Kurses. Es geht nicht nur um den Ausbau eines theoretischen Weltverständnisses, das sonst häufig im Zentrum steht. Es geht um die Erweiterung der praxisbezogenen Fähigkeiten und Fertigkeiten, mit denen wir unsere Umwelt selbst gestalten, erweitern und reparieren können. Jeder neue Lerninhalt soll im Kontext der Fragestellung \enquote{Wozu ist das nützlich?} oder \enquote{Was kann man damit anfangen?} präsentiert werden. Diese Grundhaltung knüpft an die sogenannte \emph{Maker-}Bewegung an - eine ständig größer werdende Gemeinschaft von Menschen, die neue Werkzeuge wie den Arduino, 3D-Drucker und Laser-Cutter nutzen, um selbst Dinge zu erschaffen, Dinge mit selbst ausgedachten Features zu erweitern (zu hacken) oder Dinge zu reparieren, die ganz im Sinne des Herstellers viel zu früh kaputt gegangen sind. Dabei gab es \enquote{Maker}, also Bastler, schon immer, allerdings öffnen die neuen Werkzeuge und insbesondere der Arduino Möglichkeiten, die es früher nicht gab. Dabei gehört es zum Konzept des Bastelns, dass man nicht immer alle Dinge bis ins Detail versteht, sondern sich traut, Dinge auszuprobieren und nicht zu enttäuscht zu sein, wenn diese Versuche daneben gehen. Innerhalb eines schulischen Kurses, in dem wir nicht mit selbst bezahlten Materialien experimentieren, sind diesem Ansatz Grenzen gesetzt, allerdings werden wir so weit wie möglich auch frei experimentieren. Natürlich soll das Verständnis dessen, was man da gebaut hat, nicht auf der Strecke bleiben, denn erst dieses Verständnis ermöglicht es systematisch, Fehler zu beseitigen oder neue Funktionen in vorhandene Schaltungen zu integrieren.

\section{Weitere Informationen: Bücher und Internet}

Wenn man sein Interesse und seine Freude am Basteln entdeckt hat, will man häufig weitere Informationen haben, weitere Projekte begutachten oder weitere Ideen finden, die die eigene Kreativität wiederum beflügeln. Die große Arduino-Gemeinschaft ist im Internet sehr aktiv und stellt zahlreiche Informationen bereit. Die im folgenden genannten Quellen boten mir selbst einen guten Einstieg und ich ziehe sie bei Fragen immer noch gerne heran.

Die wichtigste Anlaufstelle ist die Projekthomepage zum Arduino (\url{www.arduino.cc}). Dort finden sich nicht nur viele Informationen rund um den Arduino, sondern auch kreative Projektvorstellungen.

Weitere Projekte mit fertigen Anleitungen zum Nachbauen finden sich auf Instructables (\url{http://www.instructables.com/}). Auch Youtube stellt zahlreiche Videos von stolzen Makern bereit, die ihr Projekt vorstellen. Ein Beispiel für einen professionell betriebenen Kanal zu Elektronik ist \emph{GreatScott!} (\url{https://www.youtube.com/user/greatscottlab}). Ein weiterer professionell betriebener Kanal mit vielen guten Bastelanleitungen ist \emph{bitluni's lab} (\url{https://www.youtube.com/user/bitlunislab}). Viele gute deutschsprachige Anleitungen zum Arduino findet man bei \emph{Makerblog.at - Arduino \& Co} (\url{https://www.youtube.com/user/makerblogAT}). Nicht auf Youtube, sondern auf ihrer eigenen Homepage bieten die Leute von \emph{Funduino} (\url{https://funduino.de/}) ebenfalls gute Anleitungen für Einsteiger an.

Wer ein Buch bevorzugt, findet in dem Buch \emph{Arduino Workshops} von John Boxall einen sehr guten Ratgeber, der ganz ähnlich zu diesem Skript schrittweise und anhand von konkreten Projekten in die Welt des Arduino einführt.

Ein umfangreiches und ebenfalls praxisbezogenes Nachschlagewerk ist das Buch \emph{Arduino Praxiseinstieg} von Thomas Brühlmann. Es ist gut sortiert und bietet eine schnelle Hilfe für die meisten Fragen rund um den Arduino.

Alle genannten Quellen haben allerdings gemeinsam, dass sie mit der textbasierten Arduino - Programmierumgebung arbeiten und daher einen etwas komplexeren Einstieg bieten als dieses Skript.

\clearpage