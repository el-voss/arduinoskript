%Grundlegendes: Schriftarten, deutsche Silbentrennung, Umlaute,...
\usepackage[utf8]{inputenc} 
\usepackage[T1]{fontenc}
\usepackage[ngerman]{babel}
\usepackage{lmodern}
\renewcommand*\familydefault{\sfdefault} %% Only if the base font of the document is to be sans serif
%\usepackage{times}
%\usepackage{bookman}
%\usepackage{palatino}

% fuer Stichwortverzeichnis
%\usepackage{makeidx}
% Stichwortverzeichnis erstellen
%\makeindex

%Kolumnentitel
\usepackage{fancyhdr}
\pagestyle{fancy}
% Definition der Kolumnentitel für den fancy-Style
\fancyhead{} %clear all header fields
\fancyhead[RO]{ %
	\iftoggle{FOLIE}{
		% keine Seitenzahl auf Folien
	}{%
		\begin{tikzpicture}[]
		\fill [CadetBlue!70!green] (0,0) circle (0.25cm); %ursprünglich gray!50!white
		\fill [CadetBlue!70!green] (-0.25,0) rectangle (0.25,-0.3);
		\draw [thick, gray!50!white] (-0.1,-0.3) -- (-0.1,-0.4);
		\draw [thick, gray!50!white] (0.1,-0.3) -- (0.1,-0.5);
		\node at (0,-0.05) {\sffamily \bfseries\thepage};
		\end{tikzpicture}
	}
} %Seitenzahl auf ungeraden (odd) Seiten rechts, auf geraden (even) links (Titelseite zählt mit)
\fancyhead[LE]{%
	\iftoggle{FOLIE}{%
		% keine Seitenzahl auf Folien
	}{%
		\begin{tikzpicture}[]
		\fill[CadetBlue!70!green] (0,0) circle (0.25cm);
		\fill [CadetBlue!70!green] (-0.25,0) rectangle (0.25,-0.3);
		\draw [thick, gray!50!white] (-0.1,-0.3) -- (-0.1,-0.5);
		\draw [thick, gray!50!white] (0.1,-0.3) -- (0.1,-0.4);
		\node at (0,-0.05) {\sffamily \bfseries\thepage};
		\end{tikzpicture}
	}
}
\fancyhead[LO]{%
	\iftoggle{FOLIE}{%
		\href{https://creativecommons.org/licenses/by-nc-sa/4.0/deed.de}{\includegraphics[width=0.12\textwidth]{../pics/cc-by-nc-sa.png}~Voß}
	}{%
		\href{https://creativecommons.org/licenses/by-nc-sa/4.0/deed.de}{\includegraphics[width=0.12\textwidth]{./pics/cc-by-nc-sa.png}~Voß}
	}
}
\fancyhead[RE]{%
	\iftoggle{FOLIE}{%
		\href{https://creativecommons.org/licenses/by-nc-sa/4.0/deed.de}{Voß~\includegraphics[width=0.12\textwidth]{../pics/cc-by-nc-sa.png}}	
	}{%
		\href{https://creativecommons.org/licenses/by-nc-sa/4.0/deed.de}{Voß~\includegraphics[width=0.12\textwidth]{./pics/cc-by-nc-sa.png}}
	}
}
\fancyhead[C]{\sffamily Arduino Skript}
\fancyfoot{} %clear all footer fields
% Die ersten Seiten eines Kapitels stellen automatisch den Seitenstil auf plain um. Daher wird der Seitenstil plain hier umdefiniert
\fancypagestyle{plain}{%
	\fancyhf{} %clear header und footer
	\renewcommand{\headrulewidth}{0pt}
	\renewcommand{\footrulewidth}{0pt}
}
\renewcommand{\headwidth}{\textwidth}



%Schriften/Mathe
\usepackage{amsmath}
\usepackage{amssymb}
\usepackage{amsfonts}
\usepackage{amsthm}

%Einheiten
\usepackage[separate-uncertainty=true]{siunitx} %Angabe im Befehl durch \SI{Wert(Fehler)}{Einheit}, Ausgabe als (Wert \pm Fehler) Einheit
\sisetup{locale=DE}

%Mehrspaltige Listen
\usepackage{multicol}

%Bilder
\usepackage[table,svgnames]{xcolor}
\usepackage{graphicx}
\usepackage{float}
\usepackage[footnotesize, indention=0cm, labelsep=space, textformat=simple]{caption}
\captionsetup[wrapfigure]{name=\textbf{B}} %Die Bildunterschriften für alle wrapfigures bekommen den Namen B. statt Abbildung
\captionsetup[figure]{name=\textbf{B}}

\usepackage{pgf,tikz,pgfplots}
\usepackage{mathrsfs}
\usetikzlibrary{arrows,arrows.meta}
\usetikzlibrary{intersections}
\usetikzlibrary{shadows}
\usetikzlibrary{automata, positioning}
\usepackage{calc} % ermöglicht es, für Längen u.ä. Rechnungen anzugeben wie \marginparsep+\marginparwidth
\usepackage{wrapfig}
\usepackage{eso-pic}
\usepackage{subfig}

%Tabellen
\usepackage{booktabs} % Basispaket für Tabellen
\usepackage{array} % Basispaket für Tabellen
\usepackage{tabu} % Hauptpaket, um Tabellen flexibel zu erstellen -> diese Dokumentation lesen // einige Befehle lassen sich nicht richtig benutzen und ich weiß nicht, woran das liegt...


%Rahmen
\usepackage{tcolorbox}
\tcbuselibrary{skins}
\tcbuselibrary{breakable}
		
%einfachere Anpassung von Listen
\usepackage{enumitem} 
	
%Fußnoten
\usepackage[bottom,hang]{footmisc}

%erweiterte Schleifen und Abfragen
\usepackage{xargs} %für mehrere optionale Argumente in eigenen Makros
\usepackage{ifthen} %für den whiledo Befehl

%Zitate
\usepackage{csquotes}


%Zum Einfügen von pdf Dokumenten mit \includepdf[options]{datei.pdf}
\usepackage[final]{pdfpages}
\includepdfset{pages=-,noautoscale}
	
\usepackage[twoside, top=2cm, outer=2.6cm, inner=2.6cm, % inner und outer sind aus irgendeinem Grund vertauscht
marginparwidth=2cm, marginparsep=0.3cm,% Die Breite für die Marginalien (Randbemerkungen) auf der rechten Seite
bottom=2cm, footskip=24pt, %Abstand zwischen Textboden und Fußzeilenboden
includefoot, includehead%
]{geometry}
\usepackage[strict]{changepage} % für lokales Überschreiben der Maße, damit man in den Rand schreiben kann (bei breiten Abbildungen z.B.)


\usepackage{setspace}
\usepackage{listings} %Zur Darstellung von Code
\lstset{backgroundcolor=\color{gray!15}, keywordstyle=\bfseries, commentstyle=\color{gray!80}, basicstyle=\ttfamily, identifierstyle=, stringstyle=\color{red!60!yellow}, emphstyle=\color{blue!60!green}, numbers=left, numberstyle=\tiny, stepnumber=2, numbersep=5pt}
\lstdefinestyle{Arduino}{%
	morekeywords={Serial, Servo},% define keywords
	morecomment=[l]{//},%             treat // as comments
	morecomment=[s]{/*}{*/},%         define /* ... */ comments
	emph={setup, loop, digitalWrite, digitalRead, analogWrite, analogRead, delay, print, println, pinMode},%        keywords to emphasize
	morestring=[b]",
}
\usepackage{blindtext}

% Einfügen von Meta-Daten in das PDF-Dokument
%\usepackage{hyperxmp}
\usepackage{xmpincl}

\usepackage[unicode]{hyperref}

\hypersetup{%
colorlinks=true, linkcolor=red!60!black,%Farbe für normale LInks
citecolor=green!60!black,%Farbe für Bibliographie-Links
filecolor=blue!60!black, %Farbe für Links zu lokalen Dateien
urlcolor=magenta!60!black,%Farbe für Links zu Internetadressen
pdfhighlight=/O,%Links werden gehighlighted, wenn man darauf geht, indem sie unterstrichen (outlined) werden
pdfnewwindow=true,%Links zu PDF-Dokumenten werden in neuem Fenster geöffnet
%XMP-Daten:
pdfauthor={Sebastian Voß},%
pdflang={de},%
pdftitle={Wahlpflichtfach Arduino},%
pdfsubject={Skript zur Einführung in Elektronik und Programmierung mit dem Arduino},%
%pdfcopyright={Dieses Werk ist lizenziert unter einer Creative Commons Namensnennung - Nicht-kommerziell - Weitergabe unter gleichen Bedingungen 4.0 International Lizenz},%
%pdflicenseurl={http://creativecommons.org/licenses/by-nc-sa/4.0/},%
%pdfurl={https://el-voss.de}
}


%Einfügen von Dateien (zusätzliche Arbeitsblätter, Folien,...)
\usepackage{attachfile2}
\attachfilesetup{author=Voß, color=0 0 .4}
